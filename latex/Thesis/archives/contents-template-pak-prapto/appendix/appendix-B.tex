\chapter*{Lampiran B}

Penulisan referensi mengikuti aturan standar yang sudah ditentukan. Untuk 
internasionalisasi DTETI, maka penulisan referensi akan mengikuti standar yang 
ditetapkan oleh IEEE (\textit{International Electronics and Electrical Engineers}). Aturan penulisan ini bisa diunduh di \textcolor{blue}{http://www.ieee.org/documents/ieeecitationref.pdf}. Gunakan Mendeley sebagai \textit{reference manager} dan \textit{export} data ke format Bibtex untuk digunakan di Latex.

Berikut ini adalah sampel penulisan dalam format IEEE:

\section*{Book}

\textbf{Basic Format:}

\hangindent=4.5em
[1] J. K. Author, “Title of chapter in the book,” in Title of His Published Book, xth ed. City of Publisher, Country: Abbrev. of Publisher, year, ch. x, sec. x, pp. xxx–xxx.

\textbf{Examples:}

\hangindent=4.5em 
[1] B. Klaus and P. Horn, Robot Vision. Cambridge, MA: MIT Press, 1986.

\hangindent=4.5em 
[2] L. Stein, “Random patterns,” in Computers and You, J. S. Brake, Ed. New York: Wiley, 1994, pp. 55-70.

\hangindent=4.5em 
[3] R. L. Myer, “Parametric oscillators and nonlinear materials,” in Nonlinear Optics, vol. 4, P. G. Harper and B. S. Wherret, Eds. San Francisco, CA: Academic, 1977, pp. 47-160.

\hangindent=4.5em 
[4] M. Abramowitz and I. A. Stegun, Eds., Handbook of Mathematical Functions (Applied Mathematics Series 55). Washington, DC: NBS, 1964, pp. 32-33.

\hangindent=4.5em 
[5] E. F. Moore, “Gedanken-experiments on sequential machines,” in Automata Studies 
(Ann. of Mathematical Studies, no. 1), C. E. Shannon and J. McCarthy, Eds. Princeton, NJ: Princeton Univ. Press, 1965, pp. 129-153.

\hangindent=4.5em 
[6] Westinghouse Electric Corporation (Staff of Technology and Science, Aerospace Div.), Integrated Electronic Systems. Englewood Cliffs, NJ: Prentice-Hall, 1970.

\hangindent=4.5em 
[7] M. Gorkii, “Optimal design,” Dokl. Akad. Nauk SSSR, vol. 12, pp. 111-122, 1961 
(Transl.: in L. Pontryagin, Ed., The Mathematical Theory of Optimal Processes. New 
York: Interscience, 1962, ch. 2, sec. 3, pp. 127-135).

\hangindent=4.5em 
[8] G. O. Young, “Synthetic structure of industrial plastics,” in Plastics, vol. 3, Polymers of Hexadromicon, J. Peters, Ed., 2nd ed. New York: McGraw-Hill, 1964, pp. 15-64.

\newpage
\section*{Handbook}

\textbf{Basic Format:}

\hangindent=4.5em
[1] Name of Manual/Handbook, x ed., Abbrev. Name of Co., City of Co., Abbrev. State, year, pp. xx-xx.

\textbf{Examples:}

\hangindent=4.5em
[1] Transmission Systems for Communications, 3rd ed., Western Electric Co., 
Winston Salem, NC, 1985, pp. 44-60.

\hangindent=4.5em
[2] Motorola Semiconductor Data Manual, Motorola Semiconductor Products Inc., 
Phoenix, AZ, 1989.

\hangindent=4.5em
[3] RCA Receiving Tube Manual, Radio Corp. of America, Electronic Components and 
Devices, Harrison, NJ, Tech. Ser. RC-23, 1992.

\section*{Conference/Prosiding}

\textbf{Basic Format:}

\hangindent=4.5em
[1] J. K. Author, “Title of paper,” in Unabbreviated Name of Conf., City of Conf., Abbrev. State (if given), year, pp.xxx-xxx.

\textbf{Examples:}

\hangindent=4.5em
[1] J. K. Author [two authors: J. K. Author and A. N. Writer ] [three or more authors: J. K. Author et al.], “Title of Article,” in [Title of Conf. Record as ], [copyright year] © [IEEE or applicable copyright holder of the Conference Record]. doi: [DOI number]

\section*{Sumber Online/Internet}

\textbf{Basic Format:}

\hangindent=4.5em
[1] J. K. Author. (year, month day). Title (edition) [Type of medium]. Available: 
http://www.(URL)

\textbf{Examples:}

\hangindent=4.5em
[1] J. Jones. (1991, May 10). Networks (2nd ed.) [Online]. Available: 
http://www.atm.com

\section*{Skripsi, Tesis dan Disertasi}

\textbf{Basic Format:}

\hangindent=4.5em
[1] J. K. Author, “Title of thesis,” M.S. thesis, Abbrev. Dept., Abbrev. Univ., City of Univ., Abbrev. State, year.

\hangindent=4.5em
[2] J. K. Author, “Title of dissertation,” Ph.D. dissertation, Abbrev. Dept., Abbrev. Univ., City of Univ., Abbrev. State, year.

\textbf{Examples:}

\hangindent=4.5em
[1] J. O. Williams, “Narrow-band analyzer,” Ph.D. dissertation, Dept. Elect. Eng., 
Harvard Univ., Cambridge, MA, 1993.
\hangindent=4.5em
[2] N. Kawasaki, “Parametric study of thermal and chemical nonequilibrium nozzle 
flow,” M.S. thesis, Dept. Electron. Eng., Osaka Univ., Osaka, Japan, 1993